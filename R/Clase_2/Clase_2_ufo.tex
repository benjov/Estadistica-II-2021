\documentclass[]{article}
\usepackage{lmodern}
\usepackage{amssymb,amsmath}
\usepackage{ifxetex,ifluatex}
\usepackage{fixltx2e} % provides \textsubscript
\ifnum 0\ifxetex 1\fi\ifluatex 1\fi=0 % if pdftex
  \usepackage[T1]{fontenc}
  \usepackage[utf8]{inputenc}
\else % if luatex or xelatex
  \ifxetex
    \usepackage{mathspec}
  \else
    \usepackage{fontspec}
  \fi
  \defaultfontfeatures{Ligatures=TeX,Scale=MatchLowercase}
\fi
% use upquote if available, for straight quotes in verbatim environments
\IfFileExists{upquote.sty}{\usepackage{upquote}}{}
% use microtype if available
\IfFileExists{microtype.sty}{%
\usepackage[]{microtype}
\UseMicrotypeSet[protrusion]{basicmath} % disable protrusion for tt fonts
}{}
\PassOptionsToPackage{hyphens}{url} % url is loaded by hyperref
\usepackage[unicode=true]{hyperref}
\hypersetup{
            pdftitle={Clase de UfO's},
            pdfborder={0 0 0},
            breaklinks=true}
\urlstyle{same}  % don't use monospace font for urls
\usepackage[margin=1in]{geometry}
\usepackage{color}
\usepackage{fancyvrb}
\newcommand{\VerbBar}{|}
\newcommand{\VERB}{\Verb[commandchars=\\\{\}]}
\DefineVerbatimEnvironment{Highlighting}{Verbatim}{commandchars=\\\{\}}
% Add ',fontsize=\small' for more characters per line
\usepackage{framed}
\definecolor{shadecolor}{RGB}{248,248,248}
\newenvironment{Shaded}{\begin{snugshade}}{\end{snugshade}}
\newcommand{\KeywordTok}[1]{\textcolor[rgb]{0.13,0.29,0.53}{\textbf{#1}}}
\newcommand{\DataTypeTok}[1]{\textcolor[rgb]{0.13,0.29,0.53}{#1}}
\newcommand{\DecValTok}[1]{\textcolor[rgb]{0.00,0.00,0.81}{#1}}
\newcommand{\BaseNTok}[1]{\textcolor[rgb]{0.00,0.00,0.81}{#1}}
\newcommand{\FloatTok}[1]{\textcolor[rgb]{0.00,0.00,0.81}{#1}}
\newcommand{\ConstantTok}[1]{\textcolor[rgb]{0.00,0.00,0.00}{#1}}
\newcommand{\CharTok}[1]{\textcolor[rgb]{0.31,0.60,0.02}{#1}}
\newcommand{\SpecialCharTok}[1]{\textcolor[rgb]{0.00,0.00,0.00}{#1}}
\newcommand{\StringTok}[1]{\textcolor[rgb]{0.31,0.60,0.02}{#1}}
\newcommand{\VerbatimStringTok}[1]{\textcolor[rgb]{0.31,0.60,0.02}{#1}}
\newcommand{\SpecialStringTok}[1]{\textcolor[rgb]{0.31,0.60,0.02}{#1}}
\newcommand{\ImportTok}[1]{#1}
\newcommand{\CommentTok}[1]{\textcolor[rgb]{0.56,0.35,0.01}{\textit{#1}}}
\newcommand{\DocumentationTok}[1]{\textcolor[rgb]{0.56,0.35,0.01}{\textbf{\textit{#1}}}}
\newcommand{\AnnotationTok}[1]{\textcolor[rgb]{0.56,0.35,0.01}{\textbf{\textit{#1}}}}
\newcommand{\CommentVarTok}[1]{\textcolor[rgb]{0.56,0.35,0.01}{\textbf{\textit{#1}}}}
\newcommand{\OtherTok}[1]{\textcolor[rgb]{0.56,0.35,0.01}{#1}}
\newcommand{\FunctionTok}[1]{\textcolor[rgb]{0.00,0.00,0.00}{#1}}
\newcommand{\VariableTok}[1]{\textcolor[rgb]{0.00,0.00,0.00}{#1}}
\newcommand{\ControlFlowTok}[1]{\textcolor[rgb]{0.13,0.29,0.53}{\textbf{#1}}}
\newcommand{\OperatorTok}[1]{\textcolor[rgb]{0.81,0.36,0.00}{\textbf{#1}}}
\newcommand{\BuiltInTok}[1]{#1}
\newcommand{\ExtensionTok}[1]{#1}
\newcommand{\PreprocessorTok}[1]{\textcolor[rgb]{0.56,0.35,0.01}{\textit{#1}}}
\newcommand{\AttributeTok}[1]{\textcolor[rgb]{0.77,0.63,0.00}{#1}}
\newcommand{\RegionMarkerTok}[1]{#1}
\newcommand{\InformationTok}[1]{\textcolor[rgb]{0.56,0.35,0.01}{\textbf{\textit{#1}}}}
\newcommand{\WarningTok}[1]{\textcolor[rgb]{0.56,0.35,0.01}{\textbf{\textit{#1}}}}
\newcommand{\AlertTok}[1]{\textcolor[rgb]{0.94,0.16,0.16}{#1}}
\newcommand{\ErrorTok}[1]{\textcolor[rgb]{0.64,0.00,0.00}{\textbf{#1}}}
\newcommand{\NormalTok}[1]{#1}
\usepackage{graphicx,grffile}
\makeatletter
\def\maxwidth{\ifdim\Gin@nat@width>\linewidth\linewidth\else\Gin@nat@width\fi}
\def\maxheight{\ifdim\Gin@nat@height>\textheight\textheight\else\Gin@nat@height\fi}
\makeatother
% Scale images if necessary, so that they will not overflow the page
% margins by default, and it is still possible to overwrite the defaults
% using explicit options in \includegraphics[width, height, ...]{}
\setkeys{Gin}{width=\maxwidth,height=\maxheight,keepaspectratio}
\IfFileExists{parskip.sty}{%
\usepackage{parskip}
}{% else
\setlength{\parindent}{0pt}
\setlength{\parskip}{6pt plus 2pt minus 1pt}
}
\setlength{\emergencystretch}{3em}  % prevent overfull lines
\providecommand{\tightlist}{%
  \setlength{\itemsep}{0pt}\setlength{\parskip}{0pt}}
\setcounter{secnumdepth}{0}
% Redefines (sub)paragraphs to behave more like sections
\ifx\paragraph\undefined\else
\let\oldparagraph\paragraph
\renewcommand{\paragraph}[1]{\oldparagraph{#1}\mbox{}}
\fi
\ifx\subparagraph\undefined\else
\let\oldsubparagraph\subparagraph
\renewcommand{\subparagraph}[1]{\oldsubparagraph{#1}\mbox{}}
\fi

% set default figure placement to htbp
\makeatletter
\def\fps@figure{htbp}
\makeatother


\title{Clase de UfO's}
\author{}
\date{\vspace{-2.5em}}

\begin{document}
\maketitle

Este es un documento de R Markdown. Markdown es una sintaxis de formato
simple para la creación de documentos HTML, PDF y MS Word. Para obtener
más detalles sobre el uso de R Markdown, consulta:
\url{https://rmarkdown.rstudio.com/}

Al hacer clic en el botón ** Knit ** (``tejer''), se generará un
documento que incluye tanto el contenido como la salida de cualquier
fragmento de código R incrustado dentro del documento. Puede incrustar
un fragmento de código R como este:

\begin{Shaded}
\begin{Highlighting}[]
\CommentTok{#install.packages("rmarkdown")}
\end{Highlighting}
\end{Shaded}

También puede incrustar gráficos, por ejemplo:

\includegraphics{Clase_2_ufo_files/figure-latex/unnamed-chunk-2-1.pdf}

Tenga en cuenta que el parámetro ** echo = FALSE ** se agregó al
fragmento de código para evitar la impresión del código R que generó el
gráfico.

Para establecer opciones globales que se apliquen a cada fragmento de su
archivo o fragmento de código ** knitr::opts\_chunk\$set(echo = TRUE) **

\paragraph{Dependencies}\label{dependencies}

\begin{Shaded}
\begin{Highlighting}[]
\CommentTok{#install.packages("tidyverse")}

\KeywordTok{library}\NormalTok{(tidyverse)}
\end{Highlighting}
\end{Shaded}

\paragraph{Read and preview CSV}\label{read-and-preview-csv}

\begin{Shaded}
\begin{Highlighting}[]
\NormalTok{ufo <-}\StringTok{ }\KeywordTok{read_csv}\NormalTok{(}\StringTok{"ufo.csv"}\NormalTok{)}
\end{Highlighting}
\end{Shaded}

\begin{verbatim}
## Warning: Missing column names filled in: 'X1' [1]
\end{verbatim}

\begin{verbatim}
## 
## -- Column specification --------------------------------------------------------
## cols(
##   X1 = col_double(),
##   datetime = col_character(),
##   city = col_character(),
##   state = col_character(),
##   country = col_character(),
##   shape = col_character(),
##   `duration (seconds)` = col_double(),
##   `duration (hours/min)` = col_character(),
##   comments = col_character(),
##   `date posted` = col_character(),
##   latitude = col_double(),
##   longitude = col_double()
## )
\end{verbatim}

\begin{Shaded}
\begin{Highlighting}[]
\NormalTok{ufo }\OperatorTok\StringTok{ }\KeywordTok{head}\NormalTok{()}
\end{Highlighting}
\end{Shaded}

\begin{verbatim}
## # A tibble: 6 x 12
##      X1 datetime city  state country shape `duration (seco~ `duration (hour~
##   <dbl> <chr>    <chr> <chr> <chr>   <chr>            <dbl> <chr>           
## 1     0 10/10/1~ san ~ tx    us      cyli~             2700 45 minutes      
## 2     2 10/10/1~ kane~ hi    us      light              900 15 minutes      
## 3     3 10/10/1~ bris~ tn    us      sphe~              300 5 minutes       
## 4     4 10/10/1~ norw~ ct    us      disk              1200 20 minutes      
## 5     5 10/10/1~ pell~ al    us      disk               180 3  minutes      
## 6     6 10/10/1~ live~ fl    us      disk               120 several minutes 
## # ... with 4 more variables: comments <chr>, `date posted` <chr>,
## #   latitude <dbl>, longitude <dbl>
\end{verbatim}

El operador pipeline \%\textgreater{}\% es útil para concatenar
múltiples dplyr operaciones.

\paragraph{El número total de avistamientos de
ovnis}\label{el-nuxfamero-total-de-avistamientos-de-ovnis}

\begin{Shaded}
\begin{Highlighting}[]
\NormalTok{ufo.count <-}\StringTok{ }\NormalTok{ufo }\OperatorTok\StringTok{ }\KeywordTok{count}\NormalTok{()}

\KeywordTok{paste}\NormalTok{(}\StringTok{"Se localizaron "}\NormalTok{, ufo.count, }\StringTok{"avistamientos de ovnis"}\NormalTok{)}
\end{Highlighting}
\end{Shaded}

\begin{verbatim}
## [1] "Se localizaron  66515 avistamientos de ovnis"
\end{verbatim}

\paragraph{El número y la lista de estados, provincias y
territorios}\label{el-nuxfamero-y-la-lista-de-estados-provincias-y-territorios}

\begin{Shaded}
\begin{Highlighting}[]
\NormalTok{ufo}\OperatorTok{$}\NormalTok{state }\OperatorTok\StringTok{ }\KeywordTok{unique}\NormalTok{() }\OperatorTok\StringTok{ }\KeywordTok{length}\NormalTok{()}
\end{Highlighting}
\end{Shaded}

\begin{verbatim}
## [1] 67
\end{verbatim}

\begin{Shaded}
\begin{Highlighting}[]
\NormalTok{ufo}\OperatorTok{$}\NormalTok{state }\OperatorTok\StringTok{ }\KeywordTok{unique}\NormalTok{()}
\end{Highlighting}
\end{Shaded}

\begin{verbatim}
##  [1] "tx" "hi" "tn" "ct" "al" "fl" "ca" "nc" "ny" "ky" "mi" "ma" "ks" "sc" "wa"
## [16] "co" "nh" "wi" "me" "ga" "pa" "il" "ar" "on" "mo" "oh" "in" "az" "mn" "nv"
## [31] "nf" "ne" "or" "bc" "ia" "va" "id" "nm" "nj" "mb" "wv" "ok" "ri" "nb" "vt"
## [46] "la" "pr" "ak" "ms" "ut" "md" "ab" "mt" "sk" "wy" "sd" "pq" "de" "nd" "nt"
## [61] "qc" "sa" "ns" "yk" "pe" "yt" "dc"
\end{verbatim}

\paragraph{La duración promedio del avistamiento de ovnis por
estado}\label{la-duraciuxf3n-promedio-del-avistamiento-de-ovnis-por-estado}

\begin{Shaded}
\begin{Highlighting}[]
\NormalTok{ufo }\OperatorTok\StringTok{ }
\StringTok{  }\KeywordTok{group_by}\NormalTok{(state) }\OperatorTok\StringTok{ }
\StringTok{  }\KeywordTok{summarise}\NormalTok{(}\DataTypeTok{avg.duration =} \KeywordTok{mean}\NormalTok{(}\StringTok{`}\DataTypeTok{duration (seconds)}\StringTok{`}\NormalTok{)) }\OperatorTok\StringTok{ }
\StringTok{  }\KeywordTok{arrange}\NormalTok{(}\KeywordTok{desc}\NormalTok{(avg.duration)) }
\end{Highlighting}
\end{Shaded}

\begin{verbatim}
## # A tibble: 67 x 2
##    state avg.duration
##    <chr>        <dbl>
##  1 ar         115893.
##  2 on          62892.
##  3 hi          26196.
##  4 wa          15265.
##  5 fl          14891.
##  6 la          12445.
##  7 va          10903.
##  8 ms           9206.
##  9 ga           7708.
## 10 wv           6778.
## # ... with 57 more rows
\end{verbatim}

\paragraph{El número de avistamientos de ovnis por
estado}\label{el-nuxfamero-de-avistamientos-de-ovnis-por-estado}

\begin{Shaded}
\begin{Highlighting}[]
\NormalTok{ufo }\OperatorTok\StringTok{ }
\StringTok{  }\KeywordTok{group_by}\NormalTok{(state) }\OperatorTok\StringTok{ }
\StringTok{  }\KeywordTok{summarise}\NormalTok{(}\DataTypeTok{number.sightings =} \KeywordTok{n}\NormalTok{()) }\OperatorTok\StringTok{ }
\StringTok{  }\KeywordTok{arrange}\NormalTok{(}\KeywordTok{desc}\NormalTok{(number.sightings))}
\end{Highlighting}
\end{Shaded}

\begin{verbatim}
## # A tibble: 67 x 2
##    state number.sightings
##    <chr>            <int>
##  1 ca                8683
##  2 fl                3754
##  3 wa                3709
##  4 tx                3397
##  5 ny                2915
##  6 il                2447
##  7 az                2362
##  8 pa                2319
##  9 oh                2252
## 10 mi                1781
## # ... with 57 more rows
\end{verbatim}

\paragraph{El número de avistamientos por forma de ovni
percibida}\label{el-nuxfamero-de-avistamientos-por-forma-de-ovni-percibida}

\begin{Shaded}
\begin{Highlighting}[]
\NormalTok{ufo }\OperatorTok\StringTok{ }
\StringTok{  }\KeywordTok{group_by}\NormalTok{(shape) }\OperatorTok\StringTok{ }
\StringTok{  }\KeywordTok{summarise}\NormalTok{(}\DataTypeTok{shape.count =} \KeywordTok{n}\NormalTok{()) }\OperatorTok\StringTok{ }
\StringTok{  }\KeywordTok{arrange}\NormalTok{(}\KeywordTok{desc}\NormalTok{(shape.count))}
\end{Highlighting}
\end{Shaded}

\begin{verbatim}
## # A tibble: 28 x 2
##    shape     shape.count
##    <chr>           <int>
##  1 light           14130
##  2 triangle         6817
##  3 circle           6404
##  4 fireball         5364
##  5 unknown          4774
##  6 other            4705
##  7 sphere           4552
##  8 disk             4319
##  9 oval             3160
## 10 formation        2088
## # ... with 18 more rows
\end{verbatim}

\end{document}
